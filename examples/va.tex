% the va-template.tex file contains the document setup like styling
\include{va-template}
% ---- Your work below --------------------------------------------

\begin{document}
\maketitle % Print the title
\thispagestyle{empty}
\newpage
\tableofcontents
\thispagestyle{empty}
\newpage

\section{Einleitung}
\lipsum[1] % REPLACE WITH YOUR TEXT
Here's a simple-to create timeline, like on \url{twitter.com}.
In German maths, it's called Zahlenstrahl (ray of numbers) if
used for numbers instead of dates. Mark important events.
This serves yourself as red line (Roter Faden - I lost my line!).
This example is stolen from
\href{https://stackoverflow.com/questions/217834/how-to-create-a-timeline-with-latex}{stackoverflow.com}

Ideally, you should see a timeline here.
If not showing up yet, give me some time.
Probably KOMA thing - don't ask.
I will read warnings. I will read LATEX WARNINGS.

\begin{chronology}[5]{1983}{2010}{3ex}[\textwidth]
\event{1984}{one}
\event[1985]{1986}{two}
\event{\decimaldate{25}{12}{2001}}{three}
\end{chronology}


\section{Allgemein Wissenswertes zum Thema X}
\lipsum[2-4] % REPLACE WITH YOUR TEXT

\section{Nebenaspekte und Schauplätze / Previous work}
\lipsum[5-6] % REPLACE WITH YOUR TEXT

\section{Meine STORY}
\lipsum[7-8] % REPLACE WITH YOUR TEXT
Here's an example outline:
\begin{outline}[enumerate]
  \1 Introduction
  \1 Background
  \1 First major point
    \2 subs if any
  \1 Second and further points
  \1 Opposing views
    \2 ack em
    \2 provide response
  \1 Conclusion
    \2 restate thesis
    \2 recommended steps
    \2 why important
\end{outline}
\lipsum[9-11] % REPLACE WITH YOUR TEXT
\paragraph{Source code} can be set using using \verb|\begin{lstlisting}[language=bash]| ...
  \begin{lstlisting}[language=bash]
  FOO="bar baz"
  echo ${FOO}
  \end{lstlisting}
  % Include source from file:
  % \lstinputlisting{../FooBar.cpp}

  or set using a default verbatim environment ...
  \begin{verbatim}
  FOO="bar baz"
  echo ${FOO}
  \end{verbatim}

\section{Weitere Tätigkeiten, gesammelte Erkenntnisse}
	\lipsum[12] % REPLACE WITH YOUR TEXT

	\subsection{A subsection}
	  Foo bar baz ``bli bla blu'' blip bing bong, use \verb|\citep{...bibref...}| for citations
	  using parantheses like \citep{DBLP:books/aw/Knuth73}.
	  \textbf{Bold font} can be set using \verb|\textbf{...}|.
	  For in-text-citations, to name an author like \textcite{DBLP:books/aw/Knuth73},
	  use \verb|\textcite{...bibref...}|. Ever wanted to
	  write \texttt{rm -rf /}? Do it using \verb|\texttt{...}| for normal text
	  or using \verb!\verb|...|! for \LaTeX\ commands. Enforce line breaks
	  using double back slashes like here.
	  \\
	  Use \verb|\textcolor{colorname}{text...}|
	  to write for example in \textcolor{blue}{something} color.
	  \\
	  \textit{Italic Text}: Use \verb|\textit{...}|.
	  Good old \textsc{Kapitälchen} - what for? Emphasize using \verb|\emph{...}|: \emph{it is important}!
	  Does \citeauthor{DBLP:books/aw/Knuth73}'s quote inspire you?
	  \\
	  Use \verb|\citeauthor{...bibref...}|.
	  \texttt{citep} creates same as \texttt{parencite}: \citep{DBLP:books/aw/Knuth73}.
	  \lipsum[13] % REPLACE WITH YOUR TEXT

  	\subsection{Another subsection}
    	\lipsum[1] % REPLACE WITH YOUR TEXT

\section{Ausklang mit Medien - Fotos, Statistiken, Gerichte...}
\lipsum[14] % REPLACE WITH YOUR TEXT
\begin{figure}[H]
	\centering
	\includegraphics[width=70mm]{relevance-tree_map-mindmap.png}
	\caption{This is an example mindmap or relevance tree, created using dotty from GraphViz}
\end{figure}
\lipsum[15-17] % REPLACE WITH YOUR TEXT
Two images side by side example ... May come in handy.
\begin{figure}[H]
	\centering
	\begin{subfigure}{.5\textwidth}
	  \centering
	  \includegraphics[width=.9\linewidth]{example-unesco.jpg}
	  \caption{UNESCO example 1}
	  \label{fig:sub1}
	\end{subfigure}%
	\begin{subfigure}{.5\textwidth}
	  \centering
	  \includegraphics[width=.9\linewidth]{example-unesco.jpg}
	  \caption{Same image for illustration}
	  \label{fig:sub2}
	\end{subfigure}
	\caption{See, two images side by side. Cool?!}
	\label{fig:test}
\end{figure}


\section{Schlusswort - Conclusions}
\lipsum[18] % REPLACE WITH YOUR TEXT
Look, we know this\citep{DBLP:books/aw/Knuth73} and that.
And as Knuth told us before\citep{DBLP:books/daglib/0030428},
we have proof for facts\citep{DBLP:journals/cacm/KnuthS21}.

\section{Persönlicher Rückblick - Ausblick (``future work'')}
\lipsum[19] % REPLACE WITH YOUR TEXT


%% The rest is called Appendix and numbered as such, usually
%% https://latex-tutorial.com/latex-appendix/
\appendix

\section{Arbeitstagebuch - Diary of work}
FYI - Taken from \url{https://en.wikibooks.org/wiki/LaTeX/Tables}

\begin{tabular}{ | l | l | l | p{5cm} |}
  \hline
  Day & Min Temp & Max Temp & Summary \\ \hline
  Monday & 11C & 22C & A clear day with lots of sunshine.
  However, the strong breeze will bring down the temperatures. \\ \hline
  Tuesday & 9C & 19C & Cloudy with rain, across many northern regions. Clear spells
  across most of Scotland and Northern Ireland,
  but rain reaching the far northwest. \\ \hline
  Wednesday & 10C & 21C & Rain will still linger for the morning.
  Conditions will improve by early afternoon and continue
  throughout the evening. \\
  \hline
\end{tabular}

\section{Literaturverzeichnis}
\printbibliography % This will insert the BIBER magic here #References #Citationns

\section{Abbildungsverzeichnis} % Remind me to complete this #FIXME :)
%\thispagestyle{empty}
\listoffigures
% \listoftables - Overview of all TABLES

\section{Projektbeschreibung}
\lipsum[11] % REPLACE WITH YOUR TEXT



\end{document}
