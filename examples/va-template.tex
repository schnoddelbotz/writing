%%%%%%%%%%%%%%%%%%%%%%%%%%%%%%%%%%%%%%%%%%%%%%%%%%%%%%%%
% Example / Template for Zurich VA - Vertiefungsarbeit
%
% Based on:
%  Short Sectioned Assignment
%  http://www.LaTeXTemplates.com
%  Original author:
%  Frits Wenneker (http://www.howtotex.com)
%  License:
%  CC BY-NC-SA 3.0 (http://creativecommons.org/licenses/by-nc-sa/3.0/)
% ... and SOME hints, see URLs below
%%%%%%%%%%%%%%%%%%%%%%%%%%%%%%%%%%%%%%%%%
% plus
% https://github.com/psy-q/uol-msc-latex-template/blob/master/dissertation.tex
% PLUS
% tons of tiny tweaks, based on hints found on tex.stackexchange.com and others
% SEE BOTTOM OF THIS DOCUMENT FOR ALL HELPFUL LINKS THAT HELPED ME CREATING THIS.
% (Originally for https://twitter.com/LivUni ... but now simplified for this purpose)



%----------------------------------------------------------------------------------------
% BASIC PAGE SETUP

% A4 paper and 11pt font size
\documentclass[paper=a4, fontsize=12pt]{scrartcl}
\usepackage{geometry}   % borders e.g. [top=1.8cm, bottom=2.54cm, left=2.25cm, right=2.25cm]
\usepackage[utf8]{inputenc}
\usepackage{graphicx}  % use includegraphics to import jpegs
\usepackage[ngerman]{babel} % ngerman is neue deutsche rechtschreibung
\usepackage{pdfcomment} % permits inclusion of pdf comments (support in viewers varies)
\usepackage{amsmath,amsfonts,amsthm} % Math packages - to typeset mathematical formulas
\usepackage{lipsum}     % Used for inserting dummy 'Lorem ipsum' text into the template
\usepackage{sectsty}    % Allows customizing section commands
\usepackage{color}      % I have no idea
\usepackage{parskip}    % http://tex.stackexchange.com/questions/40429/how-to-use-the-parskip-package-space-in-between-paragraphs

% definition of custom color NAMES for later use - provide their RGB values here
\usepackage[usenames,dvipsnames,svgnames,table]{xcolor}
\definecolor{listingBgColor}{rgb}{1,1,1}
\definecolor{UOLBlue}{rgb}{0,0.03,0.25}
\definecolor{DarkBlue}{rgb}{0.05,0.05,0.35}
\definecolor{DarkRed}{rgb}{0.45,0.03,0.2}
\definecolor{LightGray}{rgb}{0.97,0.97,0.97}

% http://stackoverflow.com/questions/1012799/latex-prevent-line-break-in-a-span-of-text
\def\myurl{\hfil\penalty 100 \hfilneg \hbox}

% drawing sequence diagrams - google tikz uml and yes, UML sequence diagrams ;)
% \usepackage{tikz} % \usepackage{tikz-uml} % \usepackage{pgf-umlsd} %
% \usepgflibrary{arrows} % for pgf-umlsd

%% OUTLINE configuration.
%% Only here as howto should you wish to TYPSET your outline.
%% Use this if you are requested to SUBMIT such outline. Otherwise draft it.
%% http://tex.stackexchange.com/questions/12279/outline-of-style-i-a-i-a-1
%% adjusted to match https://owl.english.purdue.edu/owl/resource/544/03
\usepackage{outlines}
\usepackage{enumitem}
%% See outlines latex package docs if needed - you can control numbering per depth/level
\setenumerate[1]{label=\Roman*.}
\setenumerate[2]{label=\Alph*.}
\setenumerate[3]{label=\arabic*.}
\setenumerate[4]{label=\alph*.}

%% HTTP / Web-Links formatting
%\usepackage{url}  % This makes \url work
%\urlstyle{same}   % http://tex.stackexchange.com/questions/667/typeset-url-in-a-non-typewriter-font
\usepackage[]{hyperref}
\hypersetup{
    backref=false,
    colorlinks=true,
    citecolor=gray,
    linkcolor=gray,
    urlcolor=gray,
}
% \usepackage[hidelinks]{hyperref}

%----------------------------------------------------------------------------------------
% Source code listings
% \begin{lstlisting}[language=html] .... \end{lstlisting}
% https://en.wikibooks.org/wiki/LaTeX/Source_Code_Listings
\usepackage{listings}
\usepackage{textcomp}
\lstset{
    backgroundcolor=\color{listingBgColor},
    tabsize=2,
    rulecolor=\color[rgb]{0.1,0.1,0.1},
    numberstyle=\tiny\color{black},
    upquote=true,
    aboveskip={1.5\baselineskip},
    columns=fixed,
    showstringspaces=false,
    extendedchars=true,
    breaklines=true,
    prebreak = \raisebox{0ex}[0ex][0ex]{\ensuremath{\hookleftarrow}},
    frame=single,
    showtabs=false,
    showspaces=false,
    keepspaces=true,
    showstringspaces=false,
    identifierstyle=\ttfamily,
    %basicstyle=\scriptsize,
    %captionpos=b, % caption position; b for bottom; default top
    basicstyle=\footnotesize\ttfamily,
    keywordstyle=\color[rgb]{0,0,1},
    commentstyle=\color[rgb]{0.133,0.545,0.133},
    stringstyle=\color[rgb]{0.627,0.126,0.941},
}

% for pseudo code -- https://en.wikibooks.org/wiki/LaTeX/Algorithms
\usepackage{algorithmicx}
\usepackage{algpseudocode}
\usepackage{algorithm}

% allow manual caption/figure numbering
% http://tex.stackexchange.com/questions/199207/no-caption-number-for-figures-and-tables
\usepackage{caption}

% side-by-side pictures
% http://tex.stackexchange.com/questions/37581/latex-figures-side-by-side
\usepackage{subcaption}

% hierarchical diagrams using forest
% http://tex.stackexchange.com/questions/191001/how-to-draw-a-hierarchical-diagram-in-tikz
% \usepackage{forest}
% \usetikzlibrary{arrows.meta, shapes.geometric, calc, shadows}
% \colorlet{mygreen}{green!75!black}

%%% USE-CASE DESCRIPTIONS (UML / UML2)
%%% http://tex.stackexchange.com/questions/10293/latex-template-for-use-cases

% \usepackage{booktabs} % ???

%----------------------------------------------------------------------------------------
% tweakable style: header colors, document font (uncomment for Arial-Like)

% colorize headers?
%\chapterfont{\color{LightGray}}  % sets colour of chapters
%\sectionfont{\color{DarkRed}}
%\setkomafont{disposition}{\color{UOLBlue}\bfseries}

% Document Font - UoL Master Thesis requires sans serif
%\renewcommand{\familydefault}{\sfdefault}
%\usepackage{helvet}

%----------------------------------------------------------------------------------------
% Bibliography settings - The list of references / citations

\usepackage{fourier} % Use the Adobe Utopia font for the document

%%% BIBER works with your .bib files to include bibliography in final doc
%%% There are TONS of citation styles (almost per university)
%%% For HOMEWORK-like stuff, select one you like ;) - Generic teachers know ONE if any.
\usepackage[
    backend=biber,
    style=authoryear,%-ibid,
    dashed=false, % re-print recurring author names in bibliography
    citestyle=authoryear,
    %bibstyle=apalike,
    sortcites=false,
    block=space,
    useprefix=true, % de.. http://tex.stackexchange.com/questions/134258/harvard-style-bibliography-with-biblatex-almost-but-not-quite
    sortlocale=de_DE,
    natbib=true,
    url=true,
    maxcitenames=3,
    maxbibnames=99,  % show all authors in bib (not "et al") #SaneDefaults ?!
    firstinits=true, % use initials in bibliography
    uniquename=false, % http://tex.stackexchange.com/questions/65747/citation-style-in-biblatex-1-get-rid-of-first-names-2-remove-comma-before
    uniquelist=false,
    doi=true,
    eprint=false
]{biblatex}
%% The name of the file that contains all our refrences (separately. So we just reference them in this file.)
\addbibresource{va-bibliography.bib}

\renewcommand*{\finalandcomma}{} %% do not put comma before final author's "and" / "&" #SaneDefaults
% http://tex.stackexchange.com/questions/1554/biblatex-displaying-all-authors-of-multi-author-works-in-the-bibliography

% attach files to pdfs -- http://tex.stackexchange.com/questions/94811/attaching-file-into-a-pdf-with-pdflatex-will-crash-adobe-reader
% For including our malware payload. Try Eicar ;) Then submit to TurnitIn. Do not use real malware please.
\usepackage{attachfile}

% http://tex.stackexchange.com/questions/76262/biblatex-format-for-online-sources


%----------------------------------------------------------------------------------------
% tweak headers style and numbering

\usepackage{fancyhdr} % Custom headers and footers
\pagestyle{fancyplain} % Makes all pages in the document conform to the custom headers and footers
%\fancyhead{} % No page header - if you want one, create it in the same way as the footers below
\fancyhead[L]{Vertiefungsarbeit}
\fancyhead[C]{PM-99999}
\fancyhead[R]{Beispiel-Vorlage VA}

\fancyfoot[L]{Donald Duck} % Empty left footer
\fancyfoot[C]{} % Empty center footer
\fancyfoot[R]{\thepage} % Page numbering for right footer
\renewcommand{\headrulewidth}{0pt} % Remove header underlines
\renewcommand{\footrulewidth}{0pt} % Remove footer underlines
\setlength{\headheight}{13.6pt} % Customize the height of the header
\numberwithin{equation}{section} % Number equations within sections (i.e. 1.1, 1.2, 2.1, 2.2 instead of 1, 2, 3, 4)
%\numberwithin{figure}{section} % Number figures within sections (i.e. 1.1, 1.2, 2.1, 2.2 instead of 1, 2, 3, 4)
\numberwithin{table}{section} % Number tables within sections (i.e. 1.1, 1.2, 2.1, 2.2 instead of 1, 2, 3, 4)
\setlength\parindent{0pt} % Removes all indentation from paragraphs - comment this line for an assignment with lots of text

% Quotations. Zitate. Also nicht Referenzen/Verweise auf andere Literatur, sondern ein 1:1 Auszug aus einem fremden Werk.
\usepackage{epigraph}
% https://www.sharelatex.com/learn/Typesetting_quotations
% Example usage:
% \epigraph{All human things are subject to decay, and when fate summons, Monarchs must obey}{\textit{Mac Flecknoe \\ John Dryden}}

%----------------------------------------------------------------------------------------
%	TITLE SECTION

% Create horizontal rule command with 1 argument of height
\newcommand{\horrule}[1]{\rule{\linewidth}{#1}}

% Your university, school and/or department name(s)
\title{
    \normalfont \normalsize
    \textsc{Stadt Zürich} \\ [25pt]
    \horrule{0.5pt} \\[0.4cm] % Thin top horizontal rule
    \huge Beispiel-Vorlage für eine Vertiefungsarbeit \\ % The assignment title (to be defined per doc)
    \horrule{2pt} \\[0.5cm] % Thick bottom horizontal rule
}

% Mail/Link author?
\author{Donald Duck}

% Today's date or a custom date
\date{\normalsize\today}
% To date back a assignment (re-submission or whatever):
% \date{Sunday 18\textsuperscript{th} January, 2016}


% allow multi-line comments using \begin{comment} ... \end{comment}
% Imagine you could do this in Python ;) Or Perl?
\usepackage{verbatim}


% LINKS THAT MADE THIS HAPPEN

% look into standard.bbx
% http://tex.stackexchange.com/questions/176297/automatically-highlight-undefined-references
% http://tex.stackexchange.com/questions/8351/what-do-makeatletter-and-makeatother-do
% https://www.ctan.org/pkg/csquotes?lang=en --- consider use?
% http://tex.stackexchange.com/questions/180986/biblatex-no-period-after-book-and-collection-titles
% http://www.khirevich.com/latex/biblatex/
% http://tex.stackexchange.com/questions/12806/guidelines-for-customizing-biblatex-styles
% http://tex.stackexchange.com/questions/1492/passing-parameters-to-a-document
% http://tex.stackexchange.com/questions/3730/using-harvard-referencing-style
% http://tex.stackexchange.com/questions/102937/natbib-use-the-harvard-referencing-system
% http://tex.stackexchange.com/questions/26516/how-to-use-biber
% http://tex.stackexchange.com/questions/102662/harvard-reference-using-biblatex
% https://www.sharelatex.com/learn/Biblatex_citation_styles
% https://www.overleaf.com/latex/examples/a-simple-example-showing-how-to-create-harvard-style-referencing-in-latex/mnwzgkyvdbyy#.Vi6PoeorLdQ
% https://www.sharelatex.com/learn/Bibtex_bibliography_styles
% http://tex.stackexchange.com/questions/134258/harvard-style-bibliography-with-biblatex-almost-but-not-quite
% http://tex.stackexchange.com/questions/2095/simplest-way-to-typeset-entire-document-in-sans-serif-helvetica
% http://www.tug.dk/FontCatalogue/

% EOF
